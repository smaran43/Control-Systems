\documentclass{beamer}
\mode<presentation>
\usepackage{amsmath}
\usepackage{amssymb}
%\usepackage{advdate}
\usepackage{adjustbox}
\usepackage{subcaption}
\usepackage{enumitem}
\usepackage{multicol}
\usepackage{listings}
\usepackage{url}
\def\UrlBreaks{\do\/\do-}
\usetheme{Madrid}
\usecolortheme{crane}
\setbeamertemplate{footline}

{
  \leavevmode%
  \hbox{%
  \begin{beamercolorbox}[wd=\paperwidth,ht=0ex,dp=0ex,right]{author in head/foot}%
    \insertframenumber{} / \inserttotalframenumber\hspace*{2ex} 
  \end{beamercolorbox}}%
  \vskip0pt%
}
\setbeamertemplate{navigation symbols}{}

\providecommand{\nCr}[2]{\,^{#1}C_{#2}} % nCr
\providecommand{\nPr}[2]{\,^{#1}P_{#2}} % nPr
\providecommand{\mbf}{\mathbf}
\providecommand{\pr}[1]{\ensuremath{\Pr\left(#1\right)}}
\providecommand{\qfunc}[1]{\ensuremath{Q\left(#1\right)}}
\providecommand{\sbrak}[1]{\ensuremath{{}\left[#1\right]}}
\providecommand{\lsbrak}[1]{\ensuremath{{}\left[#1\right.}}
\providecommand{\rsbrak}[1]{\ensuremath{{}\left.#1\right]}}
\providecommand{\brak}[1]{\ensuremath{\left(#1\right)}}
\providecommand{\lbrak}[1]{\ensuremath{\left(#1\right.}}
\providecommand{\rbrak}[1]{\ensuremath{\left.#1\right)}}
\providecommand{\cbrak}[1]{\ensuremath{\left\{#1\right\}}}
\providecommand{\lcbrak}[1]{\ensuremath{\left\{#1\right.}}
\providecommand{\rcbrak}[1]{\ensuremath{\left.#1\right\}}}
\theoremstyle{remark}
\newtheorem{rem}{Remark}
\newcommand{\sgn}{\mathop{\mathrm{sgn}}}
\providecommand{\abs}[1]{\left\vert#1\right\vert}
\providecommand{\res}[1]{\Res\displaylimits_{#1}} 
\providecommand{\norm}[1]{\lVert#1\rVert}
\providecommand{\mtx}[1]{\mathbf{#1}}
\providecommand{\mean}[1]{E\left[ #1 \right]}
\providecommand{\fourier}{\overset{\mathcal{F}}{ \rightleftharpoons}}
%\providecommand{\hilbert}{\overset{\mathcal{H}}{ \rightleftharpoons}}
\providecommand{\system}{\overset{\mathcal{H}}{ \longleftrightarrow}}
	%\newcommand{\solution}[2]{\textbf{Solution:}{#1}}
%\newcommand{\solution}{\noindent \textbf{Solution: }}
\providecommand{\dec}[2]{\ensuremath{\overset{#1}{\underset{#2}{\gtrless}}}}
\newcommand{\myvec}[1]{\ensuremath{\begin{pmatrix}#1\end{pmatrix}}}
\let\vec\mathbf

\lstset{
%language=C,
frame=single, 
breaklines=true,
columns=fullflexible
}
\title{Control Systems \\ Assignment - 1}
\author{\Large Smaran Panth Kulkarni\\EE19BTECH11043 }
\date{}
\begin{document}

\begin{frame}
\titlepage
\begin{figure}
\includegraphics[width=120\coloumnwidth]{IITH.png}
\end{figure}
\end{frame}

\begin{frame}
\tableofcontents
\end{frame}
\section{Problem Statement}
\begin{frame}
\frametitle{Question - 45}
 
\Large In this chapter, we derived the transfer function of a dc motor relating the angular displacement output to the armature voltage input. Often we want to control the output torque rather than the displacement. Derive the transfer function of the motor that relates output torque to input armature voltage. 

\end{frame}


\begin{frame}
\section{Theory}
\frametitle{Theory behind the problem}
A motor is an electro-mechanical component that yields a displacement output for a voltage input, that is, a mechanical output generated by an electrical input.\\ \ \\ Now we shall derive the transfer function for one particular kind of electro-mechanical system, the 'Armature-controlled DC Servomotor'.
\begin{figure}
    \includegraphics[width=200\coloumnwidth]{circuit.png}
    \caption{schematic of motor}
\end{figure}
\end{frame}
%
\begin{frame}
In the figure a permanant magnetic field (Fixed Field) is generated by permanant magnets or a stationary magnets. The armature rotates due to torque generated by the fixed field and current i_a(t).\\ \ \\
In the motor as the conductor moves in the right angles of magnetic field a back emf gets generated. The back emf is directly proportional to the speed. Here is its expression :
\begin{equation}
    v_b(t) = K_b \frac{d \theta_m(t)}{dt}
\end{equation}
Back emf constant - K_b\\ we know that
\begin{equation}
    \omega_m(t) = \frac{d \theta_m(t)}{dt}
\end{equation}
By taking Laplace Transform of equation (1) we get
\begin{equation}
    V_b(s) = K_b. s \theta_m(s)
\end{equation}
\end{frame}
\section{Derivation}
\begin{frame}
\frametitle{Torque expression}

\begin{figure}
    \centering
    \includegraphics[width = 0.35\linewidth]{motor.png}
    \caption{Typical equivalent mechanical loading on a motor}
\end{figure}
Equivalent Moment of Inertia - J_m\\
Equivalent Viscous Damping - D_m\\
Torque in Laplacian Domain - T_m(s)\\ \ \\
Torques due to viscous force and mechanical loading are given by :
\begin{equation}
    T_m_{load}(s)= J_ms^2\theta_m(s) \ \ \ 
    T_m_{viscous}(s) = D_ms\theta_m(s)
\end{equation}
Expression for net torque is given by :
\begin{equation}
    T_m(s) = [J_ms + D_m]s\theta_m(s)
\end{equation}
\end{frame}

\begin{frame}
\frametitle{Derivation}

Resistance - R_a\\
Inductance - L_a\\
Motor Torque Constant - K_t\\
Current in Laplacian Domain - I_a(s)\\
Input Voltage in Laplacian Domain - E_a(s)\\ \ \\
By applying Kirchoff's Voltage Law we can write
\begin{equation}
    R_aI_a(s) + L_a sI_a(s) + V_b(s) = E_a(s)
\end{equation}
As torque developed is directly proportional to current we have
\begin{equation}
    T_m(s) = K_t I_a(s)
\end{equation}
By substituting equation (3) , (7) in (6) we get
\end{frame}
\begin{frame}
\begin{equation}
    \frac{T_m(s)[R_a + L_a s]}{K_t} +  K_b. s \theta_m(s) = E_a(s)
\end{equation}
Now in order to obtain the transfer function, we must separate the input and output variables. Simply we must write \theta_m(s) \ \  in  \  terms \  of \ T_m(s).\\ \ \\
Now substitute equation (5) in equation (8) to get 
\begin{equation}
    \frac{T_m(s)[R_a + L_a s]}{K_t} +  \frac{T_m(s)[K_b]}{J_ms + D_m} = E_a(s)
\end{equation}
\begin{equation}
    T_m(s)\Bigg[\frac{(R_a + L_as).(J_ms + D_m) + K_b.K_t}{K_t.(J_ms + D_m)}\Bigg] = E_a(s)
\end{equation}
In the above equation now divide both numerator and denominator on the left hand side by J_mL_a \ .

\end{frame}
\begin{frame}
    \begin{equation}
    T_m(s)\Bigg[\frac{\Big( s+ \frac{D_m}{J_m}\Big)\Big( s+ \frac{R_a}{L_a}\Big) + \frac{K_b.K_t}{J_m.L_a}}{\frac{K_t}{L_a}\Big( s + \frac{D_m}{J_m}\Big)}\Bigg] = E_a(s)
\end{equation}
Finally we can write the transfer function as :
\begin{equation}
    G(s) = \frac{T_m(s)}{E_a(s)} = \frac{\frac{K_t}{L_a}\Big( s + \frac{D_m}{J_m}\Big)}{\Big( s+ \frac{D_m}{J_m}\Big)\Big( s+ \frac{R_a}{L_a}\Big) + \frac{K_b.K_t}{J_m.L_a}}
\end{equation}
In simple way it can be written as 
\begin{equation}
    G(s) = \frac{m(s + n)}{s^2 + as + b}
\end{equation}
where m,n,a,b are constants.
\end{frame}
\begin{frame}
\frametitle{Special Case}
Now if the value of Inductance is very less compared to the value of Resistance, we can ignore it. In such case equation (9) gets transformed as
\begin{equation}
    \frac{T_m(s)[R_a]}{K_t} +  \frac{T_m(s)[K_b]}{J_ms + D_m} = E_a(s)
\end{equation}
Final expression for transfer function is :
\begin{equation}
    G(s) = \frac{\frac{K_t}{R_a}\Big( s + \frac{D_m}{J_m}\Big)}{ \Big(s  + \frac{D_m}{J_m} + \frac{K_b.K_t}{J_m.R_a}\Big)}
\end{equation}
In simple way it is :
\begin{equation}
    G(s) = \frac{K(s + \alpha)}{(s+\beta)}
\end{equation}
where K, \alpha , \ \beta \ are\  constants.
\end{frame}


\end{document}
%

%\begin{frame}
%\frametitle{Introduction}
%\framesubtitle{Literature}
%%\begin{figure}[t!]
%%    \centering
%%    \begin{subfigure}[t]{0.4\columnwidth}
%%        \centering
%%        \includegraphics[width=\columnwidth]{point_source}
%%        \caption{Single point source}
%%\label{fig3:subfig1}        
%%    \end{subfigure}%
%%    ~ 
%%    \begin{subfigure}[t]{0.4\columnwidth}
%%        \centering
%%        \includegraphics[width=\columnwidth]{pointNoPowerDist_new}
%%        \caption{SNR profile}
%%\label{fig3:subfig2}
%%    \end{subfigure}
%%  %  \caption{Average SNR for a BPP. $N=16$}
%%    \label{fig3}
%%  \end{figure}
%
%\end{frame}
%  
%
%
%%

\end{document}

